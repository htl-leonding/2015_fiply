\documentclass[FIPLY_base.tex]{subfiles}

%\author{Andreas Denkmayr}
%\date{25. Februar 2016}

\begin{document}
\subsubsection{Subfiles}
Um gleichzeitiges Arbeiten zu erleichtern, werden Subfiles verwendet.
Zusätzlich wird das Arbeiten übersichtlicher, da man sich nicht um ein langes Dokument kümmern muss, sondern Kapitel für Kapitel schreiben und kontrollieren kann.
Der schriftliche Teil der Diplomarbeit wird in das Basisdokument FIPLY\_base.tex und in eine Vielzahl von Subfiles unterteilt.

\ \\
Dazu wird das Subfiles Package in das Basisdokument eingebunden.
Dies erfolgt durch den \verb!\usepackage{subfiles}! Befehl.
Das Basisdokument enthält die Kapitelüberschriften und bindet die einzelnen Subfiles, mit einem Befehl wie \verb!\subfile{denkmayr_GitHub}!, ein.
Dieser Befehl verlinkt auf die \verb!denkmayr_GitHub.tex! Datei, die sich im selben Ordner wie das Basisdokument befindet.
Die Subfiles müssen \verb!\documentclass[FIPLY_base.tex]{subfiles}! in ihrem Header definieren.
Wird eine PDF aus dem FIPLY\_base.tex Dokument generiert, wird der Inhalt aller angeführten Subfiles in das Basisdokument eingefügt.

\newpage
\subsubsection{Quellen und Zitate}
Quellen und Zitierungen werden mithilfe des Biblatex Packages durchgeführt.
Für die verschiedenen Quellen werden Einträge in der literatur.bib Datei erstellt.

\begin{lstlisting}[language=Python,frame=none]
@Misc{bNavDrawer,
	author = {Ben Jakuben},
	title =  {How to Add a Navigation Drawer in Android},
	month =  feb,
	year =   {2016},
	url =    {http://blog.teamtreehouse.com/add-navigation-drawer-android}
}
\end{lstlisting}
Diese Einträge werden mit dem Programm JabRef verwaltet.

\ \\
\textbf{Wie wird zitiert?} \newline
Sind die jeweiligen Zitierungen vorhanden, kann im Text mit Befehlen wie 
\verb!\cite{bNavDrawer},! \newline 
\verb!\citeauthor{bNavDrawer},! \newline 
\verb!\citetitle{bNavDrawer},! \newline
\verb!\fullcite{bNavDrawer}! \newline
eine Zitierung eingefügt werden. 

\ \\

\ \\
So sehen diese Zitierungen in einem generierten PDF aus:
\begin{figure}[H]
	\includegraphics[scale=0.5]{img/Citations}
	\caption{Aussehen der Zitate im PDF}
\end{figure}
\ \\
Um dieses Ergebnis zu erreichen ist es notwendig das Programm biber.exe zu starten bevor eine PDF generiert wird.
Dieses Programm sucht und ordnet alle angeforderten Zitierungen und fügt diese in den Text ein.


%\ \\
%\cite{bNavDrawer},  
%
%\ \\
%\citeauthor{bNavDrawer},
%
%\ \\
%\citetitle{bNavDrawer}, 
%
%\ \\
%\fullcite{bNavDrawer}

\end{document}
