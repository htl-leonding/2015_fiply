\documentclass[FIPLY_base.tex]{subfiles}

%\author{all}

\begin{document}
\ \\
Die Arbeit an der Diplomarbeit hat uns viele wertvolle Lektionen erteilt.
Kommunikation ist der wichtigste Aspekte einer Gruppenarbeit. 
Regelmäßige Besprechungen unter den Teammitgliedern waren essentiel um die Entwicklung so effizient wie möglich zu gestalten.

\ \\
Protokolle der Besprechungen, ob mit den anderen Projektmitgliedern oder dem Betreuungslehrer sind extrem wichtig, um zu garantieren, dass wichtige Abmachungen nicht in Vergessenheit geraten.

\ \\
Wir stellten fest, dass die Planung des Projektes/Diplomarbeit viel wichtiger ist, als vorerst angenommen. 
Eine genaue und detailreiche Planung erleichtigt die Implementierung um ein Vielfaches.
Dennoch stellte sich heraus, dass der \grqq{}Quick and Dirty Approach\grqq{} ein produktive Arbeitshaltung für unser Projektteam darstellte.
Funktionen wurden zu Beginn nur so simple wie nötig implementiert und erst im Nachinein aufgehübscht.
Auf diese Weise wurde nur selten über das Ziel hinausgeschossen und wir als Team bleiben motiviert, da schnell Fortschritte erzielt wurden. 

\ \\
Während der engen Zusammenarbeit mit unserem Auftraggeber und Fitnessconsultant David Lindenbauer erhielten wir einen Einblick in das Fachgebiet der Fitness. Dies war eine wertvolle Erfahrung für uns und schaffte Motivation die Übungen und Trainingsmethoden der App auch im Selbstversuch anzuwenden.

\ \\
Zusammenfassend war die Diplomarbeit für jedes Teammitglied eine einzigartige Chance zur Fortbildung, eine teamgeiststärkendes Projekt und ein Riesenerfolg in allen Aspekten. 

\end{document}